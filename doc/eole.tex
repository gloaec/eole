\documentclass{beamer}

%\usertheme{Madrid}
\setbeamercovered{transparent}
\setcounter{tocdepth}{2}
\usepackage[french]{babel}
\usepackage[utf8,utf8x]{inputenc}
\usepackage[T1]{fontenc}
%\usepackage{xunicode} %Unicode extras!
%\usepackage{xltxtra}  %Fixes
%\setmainfont{CaviarDreams}
\usepackage{multicol}
%\usepackage{colortbtl}
\usepackage{graphicx}
\usepackage{verbatim}            % Pour l'insertion de fichier en mode verbatim
\usepackage{ucs}
\usepackage{tabto}
%\usecolortheme{crane}
\definecolor{UniBlue}{RGB}{83,121,180}
\definecolor{CleanWhite}{RGB}{255,255,255}
\setbeamercolor{title}{fg=UniBlue}
\setbeamercolor{frametitle}{fg=CleanWhite}
\setbeamercolor{structure}{fg=UniBlue} 
\newcommand{\eolesmall}{
    \begin{minipage}[c]{0.10\textwidth}
        \includegraphics[width=1cm]{img/logo-eole.png}
    \end{minipage}
}
\newcommand{\eolebig}{\includegraphics[width=2cm]{img/logo-eole.png}}
\newcommand{\eolefull}{\includegraphics{img/logo-eole.png}}
%\setmonofont[Scale=0.86]{Andale Mono}
%\usepackage{colortab}

\setbeamertemplate{background}{\includegraphics[width=128mm]{img/banner01.png}}

\title[]{EOLE}

\subtitle{Turnkey solution Overview}

\author[Team Author]{Cadoles Team}

\institute[E.O.L.E]{\includegraphics[width=2cm]{img/logo-eole.png}}
%\institute[Cadoles]{\includegraphics[width=2cm]{img/logo-cadoles-01.png}}

\date{\today}

\subject{Talks}

\AtBeginSubsection[]
{
  \begin{frame}<beamer>
      \frametitle{}
    \tableofcontents[currentsection,currentsubsection]
  \end{frame}
}


\logo{
    \includegraphics[width=1cm]{img/logo-ecologie.png}~
    \hspace{120pt}
    \includegraphics[width=2cm]{img/logo-cadoles-01.png}~
    \hspace{113pt}
    \includegraphics[width=1cm]{img/logo_en.jpg}~
}

\begin{document}

    % Page de titre
    \begin{frame}
        \titlepage
    \end{frame}

    \begin{frame}
	    \frametitle{The \textsc{EOLE} Project}

        \begin{itemize}
        	\item \textbf{How is it born ?}

		    The project has been initiated in 2000 with the Amon firewall. 
	    	This first piece of work is still sustained and aims at providing 
       		a shared internet access from a single gate.

        	\item \textbf{A French national scale project}

    		In 2001, \textsc{EOLE} becomes the official national education solution 
    		and is going to quickly spread around from there.

        	\item \textbf{The \textsc{EOLE} Team}

	    	In 2003, the highly skilled \textsc{EOLE} team takes other minitries 
	    	along and private customers as well. The project widely spreads and 
	    	bacomes definitely a well tested global IT solution. 
    	\end{itemize}
    \end{frame}

% Frame Historique/Versions EOLE

    \begin{frame}
	    \frametitle{Missions}

        \begin{itemize}
		    \item \textsc{EOLE} delivers turnkey solutions to setup 
            intranet/internet servers.
		    \item \textbf{\textsc{EOLE} exclusively uses and provides free softwares.}
	        \item *** Only Free software *** 
	    \end{itemize}
    \end{frame}

    \begin{frame}
	    \frametitle{\textsc{EOLE}'s Scope}

        In the ministery of education:
	    \begin{itemize}
	    	\item 15 million students
	    	\item 1,5 million of teachers, engineers, technical staff, 
            administatrators, managers\ldots
	    	\item 63 000 schools (primary, secondary and high schools)
	    	\item 14 000 educational institutions
	    \end{itemize}
	    But also in territorial authorities, Department of Ecology 
        and other institutions.
    \end{frame}

    \begin{frame}
	    \frametitle{Governance}
	    
    	\begin{itemize}
    		\item \textsc{EOLE}, a mutualized project
		
    		\item Our moral values

    		Creativity -- Quality -- Scalability

    		\item Our methodology

    		We use some of the principles of agile software development, 
    		well-known from open source communitites and service providing 
	    	companies.

    	\end{itemize}
    \end{frame}

    \begin{frame}
    	\frametitle{Our Agile Software Development Rules}

	    \begin{itemize}
	    	\item The team
		
    		People and interactions instead of waterfall processus and tools

    		\item The application

		    Priority given to the operational software instead of writing 
	    	plethora of documents
    
    		\item The collaboration

	    	Aware of the customer/user remarks instead of contract negociation
    
    		\item The acceptance of changes

	    	Welcome potential modifications instead of following the initial
    		schedule

    		\item Continuous integration

		    As soon as a task is over, the feature is immediatly available. 
		    Release often, so the software can be tested at once.
	    \end{itemize}
    \end{frame}


    \begin{frame}
	    \frametitle{Community}

		Essential for an Open Source project.
		Loosely affilated groups of disparate people working on
		free software together

		\begin{itemize}
			\item Seminars: Once a year
			\item Workshops: Once to twice a year
			\item Attendance at shows and conferences
			\item Everyday :
			\begin{itemize}
				\item Broadcast site -- Reporting -- Wiki
				\item Broadcast lists
				\item IRC Channel
			\end{itemize}
		\end{itemize}
	\end{frame}

	\begin{frame}
		\frametitle{Community}

		Essential for an Open Source project.
		Loosely affilated groups of disparate people working on
		free software together

		\begin{itemize}
			\item Seminars: Once a year
			\item Workshops: Once to twice a year
			\item Attendance at shows and conferences
			\item Everyday :
			\begin{itemize}
                \item Broadcast site -- Reporting -- Wiki
				\item Broadcast lists
				\item IRC Channel
			\end{itemize}
        \end{itemize}
    \end{frame}

\end{document}
